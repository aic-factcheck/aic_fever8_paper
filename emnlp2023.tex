% This must be in the first 5 lines to tell arXiv to use pdfLaTeX, which is strongly recommended.
\pdfoutput=1
% In particular, the hyperref package requires pdfLaTeX in order to break URLs across lines.

\documentclass[11pt]{article}

% Remove the "review" option to generate the final version.
\usepackage[review]{EMNLP2023}

% Standard package includes
\usepackage{times}
\usepackage{latexsym}
\usepackage{float}

% For proper rendering and hyphenation of words containing Latin characters (including in bib files)
\usepackage[T1]{fontenc}
% For Vietnamese characters
% \usepackage[T5]{fontenc}
% See https://www.latex-project.org/help/documentation/encguide.pdf for other character sets

% This assumes your files are encoded as UTF8
\usepackage[utf8]{inputenc}

% This is not strictly necessary, and may be commented out.
% However, it will improve the layout of the manuscript,
% and will typically save some space.
\usepackage{microtype}

% This is also not strictly necessary, and may be commented out.
% However, it will improve the aesthetics of text in
% the typewriter font.
\usepackage{inconsolata}
% includegraphics
\usepackage{graphicx}
%listings
\usepackage{listings}
\usepackage{dirtytalk}

%pandas tables
\usepackage{{booktabs}}
% Commands
\newcommand{\todo}[1]{{\color{red}\colorbox{yellow}{\textbf{TODO: }}#1}}
\newcommand{\averitec}{AVeriTeC}
\newcommand{\supp}{Supported}
\newcommand{\reff}{Refuted}
\newcommand{\nei}{Not enough evidence}
\newcommand{\conf}{Conflicting evidence/Cherrypicking}
\makeatletter
\newcommand\footnoteref[1]{\protected@xdef\@thefnmark{\ref{#1}}\@footnotemark}
\makeatother

% If the title and author information does not fit in the area allocated, uncomment the following
%
%\setlength\titlebox{<dim>}
%
% and set <dim> to something 5cm or larger.

\title{AIC CTU system at \averitec{}: Re-framing automated fact-checking as a simple RAG task}

% Author information can be set in various styles:
% For several authors from the same institution:
% \author{Author 1 \and ... \and Author n \\
%         Address line \\ ... \\ Address line}
% if the names do not fit well on one line use
%         Author 1 \\ {\bf Author 2} \\ ... \\ {\bf Author n} \\
% For authors from different institutions:
% \author{Author 1 \\ Address line \\  ... \\ Address line
%         \And  ... \And
%         Author n \\ Address line \\ ... \\ Address line}
% To start a seperate ``row'' of authors use \AND, as in
% \author{Author 1 \\ Address line \\  ... \\ Address line
%         \AND
%         Author 2 \\ Address line \\ ... \\ Address line \And
%         Author 3 \\ Address line \\ ... \\ Address line}

\author{Herbert Ullrich \\
AI Center @ CTU FEE\\
Charles Square 13\\
Prague, Czech Republic\\
\texttt{ullriher@fel.cvut.cz} \\\And
Tomáš Mlynář \\
AI Center @ CTU FEE\\
Charles Square 13\\
Prague, Czech Republic\\
\texttt{mlynatom@fel.cvut.cz} \\ \\\And
Jan Drchal \\
AI Center @ CTU FEE\\
Charles Square 13\\
Prague, Czech Republic\\
\texttt{drchajan@fel.cvut.cz} \\}

\begin{document}
{\makeatletter\acl@finalcopytrue
  \maketitle
}
\begin{abstract}
This paper describes our $3^{rd}$ place submission in the \averitec{} shared task in which we attempted to address the challenge of fact-checking using evidence retrieved from web using a simple scheme of Retrieval-Augmented Generation designed for the task.\footnote{\url{https://github.com/aic-factcheck/aic_averitec}}
\end{abstract}

%%%%%%%%%%%%%%%%%%%%%%%%%%%%%%%%%%%%
% inputs
%!TEX ROOT=../emnlp2023.tex

\section{Introduction}
\label{sec:introduction}
\todo{write}

% show figures/pipeline.png
\begin{figure}[h]
    \centering
    \includegraphics[width=0.5\textwidth]{figures/pipeline.pdf}
    \caption{Our pipelines}
    \label{fig:pipeline}
\end{figure}


%!TEX ROOT=../emnlp2023.tex

\section{System Description}
\label{sec:system}
\todo{write}
Our syste

\subsection{Retrieval Module}
To ease comparison with the baseline and other systems designed for the task, our system does not use direct internet/search-engine access for its retrieval, but a \textit{knowledge store} provided separately alongside each claim.

\subsubsection{Knowledge Stores}
Each claim's knowledge store contains pre-scraped search results for various queries that can be derived from the claim using human annotation or generative models.
The knowledge stores used with ours as well as the baseline system can be downloaded from the Averitec dataset page\footnote{\url{https://fever.ai/dataset/averitec.html}}, containing about 1000 pre-scraped \textit{documents}\footnote{\label{devsetnote} The numbers are orientational and were established using the dev knowledge store}, each consisting of $28$ sentences at median\footnoteref{devsetnote}, albeit varying wildly between documents.

To use our system in the wild, this knowledge store can be emulated using a search API such as SerpApi, or even a large document collection such as Common Crawl pruned down to similar orders of magnitude using a cheap retrieval method and the claim as a search query.

Our retrieval module then focuses on picking a set of $k$ ($k=10$ in the examples below) most appropriate document chunks to fact-check the provided claim within this knowledge store.

\subsubsection{Chunking}
Our initial experiments with the whole AVerITeC documents for the Document Retrieval step have revealed a significant weakness -- while the median document length (about 2000 characters) fits the input size of our embedding model with a generous margin, there is often a small number of documents with \textit{hundreds of thousands} characters, exceeding the 1024 input tokens with little to no coverage.

Upon further examination, these have more often than not, been PDF documents of legislature, documentation and transcription -- highly relevant sources real fact-checker would scroll through to find the relevant part to refer. 

This workflow has inspired our next approach -- to facilitate the retrieval of smaller articles as a whole and 

\subsubsection{Search-space Pruning}
While the chunking of long articles prevents information from larger documents from being lost, it makes the domain for embedding search too large.
As every claim has its own retrieval domain of tens of thousands of chunks, we seek to omit the chunks having little to no common tokens with our claim using a BM25 search for the nearest $k$ chunks, setting the $k$ to 6000 for dev and 2000 for test claims. 
This yields a reasonably-sized document store for embedding into a vectorstore, taking about 40s to compute and store within FAISS~\cite{douze2024faiss} for each test-claim using our Tesla V100 GPU.

This allows a quick and agile production of vectorstores for further querying and experimentation, motivated by the Averitec test data only being published just several days before the announced submission deadline while also keeping the resource intensity moderate for real-world applications -- if time is not of the essence, the step can be omitted.

\subsubsection{Angle-optimized embedding search}
Mixedbread~\cite{li-li-2024-aoe,emb2024mxbai}, Faiss~\cite{douze2024faiss,johnson2019billion}

\subsubsection{Looking for diversity}
While the original~\cite{averitec2024} baseline retrieved articles based on different queries to promote variety among search results, our approach omits the division of a claim to a set of different queries.
We aim to use an embedding-driven similarity search in the neighbourhood of the whole original claim not to leave any piece of information behind or introduce noise in yet another generative task along the pipeline.

Our solution is, however, prone to redundancy among search results, which we address using a reranking by the results' Maximal Marginal Relevance (MMR)~\cite{carbonell-mmr}, a metric popular for the RAG task computed as
$$\lambda \cdot \mathrm{Sim}(D_i, Q) - (1-\lambda) \cdot \max_{D_j \in S} \mathrm{Sim}(D_i, D_j)$$

In our system, we set $\lambda=0.75$ to favour relevancy rather than diversity, $k=10$ and $k_{fetch} = 40$, obtaining a set of diverse sources relevant to each claim at a fraction of cost and complexity of a query-generation driven retrieval, such as that used in~\cite{averitec2024}.

\subsection{Evidence Generation}
Rather than sampling evidence from retrieved text or QG+QA pipeline ran on the retrieved sentences, we argue that the modern LLMs offer the use of wider context, such as whole news articles.

\subsubsection{\texttt{JSON} Generation}
The current LLMs are trained very well for this, allows for very simple integration of LLM into pipeline

\subsubsection{Source Referring}
We assign a 1-based index to each of the sourced chunks and prompt the LLM to refer it as the source ID with each evidence it generates.
This has been shown quite reliable in \todo{literature}.

\subsubsection{Chain-of-thoughts Prompting}
While JSON dictionary should be order-invariant, we can actually exploit the order of outputs to make LLMS like GPT-4o output better results.

\subsubsection{Few-shot learning}
To further boost our system's evidence-generation capabilities, 



%!TEX ROOT=../emnlp2023.tex

\subsection{Classification Step}
In this section, we describe the classification component of our pipeline. First, we introduce chronologically the different approaches, the ensembles we tried, and their pros and cons. Then, we present evaluation results across different methods and various metrics. The goal of this step is to take the input claim and questions with answers provided by previous stages \todo{ref} and classify the claim into one of the four classes: \textit{Supported}, \textit{Refuted}, \textit{Not Enough Evidence}, or \textit{Conflicting Evidence/Cherrypicking} as defined by~\citealp{averitec2024}.

\subsubsection{Approaches}
\todo{Describe what was done +  why}

In the earliest stages of experimenting, we utilized the classifier from baseline provided by authors\footnote{https://huggingface.co/chenxwh/AVeriTeC}~\cite{averitec2024}. This classifier is based on the BERT~\cite{devlin-etal-2019-bert} model and was further fine-tuned on the AVeriTeC dataset~\cite{averitec2024}. It takes one claim and one question with its answer as input. The output of this encoder model (class logits) is then further processed by several if-clauses to determine the final label. To the best of our knowledge, the complete classifier outputs \textit{Not Enough Evidence} whenever the model predicts the \textit{Not Enough Evidence} or some 4th label (here, we are not sure how the model was precisely fine-tuned) for any of the (up to 10) question-answer pairs. It predicts \textit{Supported} if there is no \textit{Not Enough Evidence} prediction and at least one \textit{Supported} prediction and no \textit{Refuted} prediction. Then it predicts \textit{Refuted} if there are no \textit{Not Enough Evidence} or \textit{Supported} predictions and at least one \textit{Refuted} prediction. Otherwise, it predicts \textit{Conflicting Evidence/Cherrypicking}. Despite considerable effort, we could not understand why and how the model was trained on four labels and exact background of this, at least from our point of view, ad-hoc post-processing logic.

We argue that the post-processing logic should be changed so that \textit{Not Enough Evidence} is predicted only if there is no other label to predict. We think that, for example, if among the predictions for question-answer pairs, there is one \textit{Not Enough Evidence} prediction and the rest are \textit{Supported} predictions, the final prediction should be \textit{Supported} and not \textit{Not Enough Evidence} as in the original logic. We did implement this change (see listing~\ref{lst:post-processing}), and surprisingly, it did not improve the results as the number of \textit{Not Enough Evidence} class has fallen significantly \todo{numbers}. Despite that, we still think this change is more logical and should be used in the future.

\lstset{
    language=Python,
    basicstyle=\ttfamily\footnotesize\linespread{0.9}, % Smaller font with less spacing
    keywordstyle=\color{blue}\bfseries,
    commentstyle=\color{green!50!black}\itshape,
    stringstyle=\color{orange},
    numberstyle=\tiny\color{gray},
    numbers=left, % Line numbers on the left
    stepnumber=1, % Line numbers for every line
    numbersep=5pt, % Space between line numbers and code
    tabsize=4, % Size of tabs
    showstringspaces=false, % Don't show spaces in strings
    breaklines=true, % Line wrapping
    breakatwhitespace=true,
    frame=lines, % Add a frame around the code
    captionpos=b, % Caption at the bottom
}
\begin{figure}
    \begin{lstlisting}[language=Python, frame=single]
if has_true and has_false:
  answer = 3
elif has_true and not has_false:
  answer = 0
elif not has_true and has_false:
  answer = 1
else:
 answer = 2 #otherwise NEI
        \end{lstlisting}
    \caption{Our proposed post-processing logic} 
    \label{lst:post-processing}       
\end{figure}

Moreover, we think the model should be trained only on three labels because only three are used. We tried our fine-tuning of a newer encoder model DeBERTaV3~\cite{he2023debertav3improvingdebertausing} on only three labels, and the results were better \todo{numbers}.

The results of our preliminary experiment were not satisfactory, and we decided to try more radically different approaches, as described in the following sections.

\subsubsection*{Concatenation and Four Classes Classifier}
\label{subsubsec:concatenation}
\todo{better name}
\todo{describe, models, Concatenation, sep-concatenation, random order, four classes!, Mistral, DeBERTaV3, metion Transformers-HF}
Our first approach is to finetune a text classification model on all four classes directly. However, this means that as evidence, we must feed the model with all question-answer pairs at once to classify \textit{Conflicting Evidence/Cherrypicking} correctly. For that, we tried several options. First, we concatenated all of them using single blank space in the order they are provided in the dataset. This should work because we expect that annotators created the question-answer pairs usually in a specific order. The second option we tried is concatenation using a separator \texttt{[SEP]} token again in the original order. This option utilizes the model's knowledge about separator tokens and allows it to know exact question-answer borders. The last option we tried is to randomly select more orders of the question-answer pairs (all combinations computation is not feasible - up to $10!$ of combinations).

As models, we chose DeBERTaV3~\cite{he2023debertav3improvingdebertausing} in two variants: the original large one\footnote{https://huggingface.co/microsoft/deberta-v3-large} and one pre-finetuned on NLI tasks\footnote{https://huggingface.co/cross-encoder/nli-deberta-v3-large}, and also Mistral-7B-v0.3 model\footnote{https://huggingface.co/mistralai/Mistral-7B-v0.3} with a classification head (MistralForSequenceClassification) provided by the Huggingface Transformers library~\cite{wolf-etal-2020-transformers} that utilizes the last token. In the preliminary testing phase, the original DeBERTaV3 Large performed the best and was used in all other experimental settings.

From the approaches described above, we achieved the best results with the original order achieving 0.71 macro $F_1$ score. The separator model achieved a comparable 0.70 macro $F_1$ score, and the random order model performed worse with a 0.67 macro $F_1$ score. We provide our best DeBERTaV3 finetuned model publicly in a Huggingface repository\footnote{https://huggingface.co/ctu-aic/deberta-v3-large-AVeriTeC-nli}.

\subsubsection*{LLM Classifiers}
\todo{describe - likert motivation, GPT4o, Claude 3?}
\todo{try "opensource" LLMs? - maybe own section in an Appendix?}
To utilize the chain-of-thoughts abilities of large language models (LLMs), such as GPT4o~\cite{openai2024gpt4o} or Claude 3.5 Sonnet~\cite{anthropic2024claude35sonnet}, we tried to use them also for the classification task. Because we wanted to be able to perform ensembling (see section~\ref{subsubsec:ensembling}), we needed to output class probabilities. However, for the LLMs the math and numbers poses still a challenge~\cite{ahn-etal-2024-large}. To compensate the weaker mathematical reasoning abilities, we use the Likert scale as a proxy for the class probabilities. The Likert scale was the standard one, and the model was asked (see system prompt in Appendix~\ref{appendix_sec:llms}) to provide the Likert scale rating for each class. During postprocessing, we then transformed those to number scores\footnote{strongly disagree: -2, disagree: -1, neutral: 0, agree: 1, strongly agree: 2} and then we used the softmax function to get the wanted class probabilities.

\todo{check}
We know that our approach uses closed-source models; however, we could not utilize the open-source models due to limited time. Because we want to support the usage of open-source models, we provide the evaluation on open-source (or at least open-weights) models such as Llama 3.1~\cite{meta2024llama31}, Command~R~\cite{cohere2024commandr} in the Appendix~\ref{appendix_sec:opensource_llms}.

\subsubsection*{Ensembling}
\label{subsubsec:ensembling}
\todo{describe ensembling - average, weighted average, stacking using logreg}

Motivated by the results of each model, which were good at classifying into one of the classes, we decided to try ensembling the models. The first ensembling method we tried was simple averaging of the class probabilities of the models.

The second method is a logical extension of the first one, tuning the weights used in average and allowing us to use different weights for different models. This method was motivated by LLM classifiers, which often output exact probabilities for multiple classes. In this case, we can use the second model as a tiebreaker with a smaller weight in the average. The weights were optimized using bounded \texttt{minimize\_scalar} function from the Scipy library~\cite{2020SciPy-NMeth}.

The last method we tried was stacking using logistic regression. However, this setup classified no labels from \textit{Not Enough Evidence} and \textit{Conflicting Evidence/Cherrypicking}, and we could not achieve reasonable results. For logistic regression, we used the scikit-learn library~\cite{scikit-learn}.

\subsubsection*{Conflicting Evidence/Cherrypicking Detection}
\todo{Binary deberta with custom loss function - weighted crossentropy loss}
\todo{Bryce's idea - tf-idf + randomforests?, future works-new paper (not exactly what we are doing now, but interesting)~\cite{jaradat2024contextawaredetectioncherrypickingnews}}

During the experiments, we discovered that classifying the \textit{Conflicting Evidence/Cherrypicking} class is the most challenging task. To overcome this problem, we tried to build a binary classifier that would output one if the input claim is cherrypicked and 0 otherwise. First, we tried to use the DeBERTaV3 Large model with basic cross-entropy loss (other experimental settings were the same as in section~\ref{subsubsec:concatenation}). However, due to a high imbalance in the dataset, the model was not able to learn the classification task. To tackle this issue, we came up with a custom weighted cross-entropy loss function that would penalize the model more for misclassifying the \textit{Conflicting Evidence/Cherrypicking} class\footnote{The used weights corresponds to the ratio of classes in the training dataset: [0.3, 5]}. This approach improved slightly the performance, but it was still not usable in our pipeline.

Another approach we tried due to the lack of large training sets for cherrypicking was using a simple tf-idf representation of just the input claim and then using a random forest classifier~\footnote{In both cases, we used implementation from the Scikit-learn library~\cite{scikit-learn}}. However, this approach still did not provide satisfactory results.

While writing this system description paper, we found an interesting research by~\citet{jaradat2024contextawaredetectioncherrypickingnews} that uses a radically different approach to detect cherrypicking in newspaper articles.


\subsubsection{Classification Evaluation}
\todo{table with comparative results + comments}

\begin{table*}
    \centering
    \begin{tabular}{lrrrr}
        \toprule
        Classifier & Accuracy & $F_1$ score & Precision & Recall \\
        \midrule
        GPT4o & \textbf{0.72} & 0.46 & 0.48 & 0.47 \\
        Claude 3.5 Sonnet & 0.64 & 0.49 & 0.50 & 0.52 \\
        DeBERTa & 0.63 & 0.39 & 0.40 & 0.41 \\
        DeBERTa - random@10 & 0.65 & 0.41 & 0.41 & 0.44 \\
        $0.5*\mbox{DeBERTa}+0.5*\mbox{GPT4o}$ & 0.70 & 0.43 & 0.41 & 0.45 \\
        $0.5*\mbox{DeBERTa}+0.5*\mbox{Claude 3.5 Sonnet}$ & 0.68 & 0.47 & 0.50 & 0.49 \\
        $0.3*\mbox{DeBERTa}+0.7*\mbox{GPT4o}$ & \textbf{0.72} & 0.45 & 0.45 & 0.46 \\
        $0.3*\mbox{DeBERTa}+0.7*\mbox{Claude 3.5 Sonnet}$ & 0.66 & \textbf{0.50} & \textbf{0.51} & \textbf{0.53} \\
        $0.1*\mbox{DeBERTa}+0.9*\mbox{GPT4o}$ & \textbf{0.72} & 0.39 & 0.46 & 0.43 \\
        $0.1*\mbox{DeBERTa}+0.9*\mbox{Claude 3.5 Sonnet}$ & 0.64 & 0.49 & 0.50 & 0.54 \\

        \bottomrule
    \end{tabular} 
    \caption{Evalution of the classifiers on the development set. $F_1$, Precision and Recall are computed as macro-averages. The random@10 suffix indicates that the classifier run with 10 different random orders of question-answer pairs. GPT4o stands for the Likert classifier based on GPT-4o, Claude 3.5 Sonnet is the Likert classifier based on Claude 3.5 Sonnet, and DeBERTa is the Likert classifier based on DeBERTaV3 Large.}
    \label{tab:nli}
\end{table*}

%!TEX ROOT=../emnlp2023.tex

\section{Results and analysis}
\label{sec:results}

We examine our pipeline results using two sets of metrics -- firstly, we measure the prediction accuracy and $F_1$ over predict labels without any ablation, that is obtaining predicted labels using the predicted evidence generated on top the predicted retrieval results. 
While the retrieval module is fixed throughout the experiment (a full scheme described in section~\ref{retrieval}), various Evidence \& Label generators and classifiers are compared in Table~\ref{tab:nli}, showcasing their performance on the same sources.
The results show that if we disregard the quality of evidence, models are more or less interchangeable, without a clear winner across the board -- an ensemble of DeBERTA and Claude-3.5-Sonnet gives the best $F_1$ score, while GPT-4o scores 72\% accuracy.
\begin{table}\begin{table*}
    \centering
    \begin{tabular}{lrrrr}
        \toprule
        Classifier & Accuracy & $F_1$ score & Precision & Recall \\
        \midrule
        GPT4o & \textbf{0.72} & 0.46 & 0.48 & 0.47 \\
        Claude 3.5 Sonnet & 0.64 & 0.49 & 0.50 & 0.52 \\
        DeBERTa & 0.63 & 0.39 & 0.40 & 0.41 \\
        DeBERTa - random@10 & 0.65 & 0.41 & 0.41 & 0.44 \\
        $0.5*\mbox{DeBERTa}+0.5*\mbox{GPT4o}$ & 0.70 & 0.43 & 0.41 & 0.45 \\
        $0.5*\mbox{DeBERTa}+0.5*\mbox{Claude 3.5 Sonnet}$ & 0.68 & 0.47 & 0.50 & 0.49 \\
        $0.3*\mbox{DeBERTa}+0.7*\mbox{GPT4o}$ & \textbf{0.72} & 0.45 & 0.45 & 0.46 \\
        $0.3*\mbox{DeBERTa}+0.7*\mbox{Claude 3.5 Sonnet}$ & 0.66 & \textbf{0.50} & \textbf{0.51} & \textbf{0.53} \\
        $0.1*\mbox{DeBERTa}+0.9*\mbox{GPT4o}$ & \textbf{0.72} & 0.39 & 0.46 & 0.43 \\
        $0.1*\mbox{DeBERTa}+0.9*\mbox{Claude 3.5 Sonnet}$ & 0.64 & 0.49 & 0.50 & 0.54 \\

        \bottomrule
    \end{tabular} 
    \caption{Evalution of the classifiers on the development set. $F_1$, Precision and Recall are computed as macro-averages. The random@10 suffix indicates that the classifier run with 10 different random orders of question-answer pairs. GPT4o stands for the Likert classifier based on GPT-4o, Claude 3.5 Sonnet is the Likert classifier based on Claude 3.5 Sonnet, and DeBERTa is the Likert classifier based on DeBERTaV3 Large.}
    \label{tab:nli}
\end{table*}\end{table}

\begin{table*}\begin{table*}[h]
    \centering
    \begin{tabular}{l | c c c | c c c}
    \hline
    &\multicolumn{3}{c|}{\textbf{Dev Set Scores}} & \multicolumn{3}{c}{\textbf{Test Set Scores}}  \\
    \textbf{Pipeline Name} & \textbf{Q only} & \textbf{Q+A} & \textbf{\averitec{}} & \textbf{Q only} & \textbf{Q+A} & \textbf{AVeriTeC} \\ \hline

    
    \textbf{GPT-4o (full-featured pipeline)}      & \textbf{0.46} & \textbf{0.29} & \textbf{0.42} & \textbf{0.46} & \textbf{0.32} & \textbf{0.50}\\
    GPT-4o (simplified pipeline)         & 0.45 & 0.28 & 0.38 & 0.45 & 0.30 & 0.47 \\
    Claude-3.5-Sonnet (full-featured)             & 0.43 & 0.28 & 0.35 & 0.42 & 0.30 & 0.46 \\
    GPT-4o (with DeBERTa classification)              & 0.45 & 0.28 & 0.36 & -- & -- & --\\
    \averitec{} baseline            & 0.24 & 0.19 & 0.09 & 0.24 & 0.20 & 0.11\\
    \hline
    Llama 3.1 70B (full-featured) & 0.46 & 0.27 & 0.36 & 0.47 & 0.29 & 0.42\\
    \bottomrule
    \end{tabular}
    \caption{Comparison of Pipeline Scores on Dev and Test Sets, AVeriTeC scores are @0.25}
    \label{tab:pipeline_scores}
\end{table*}
    \end{table*}
In real world, however, the evidence quality is critical for the fact-checking task.
We therefore proceed to estimate it using the hu-METEOR evidence question score, QA score and \averitec{} score benchmarks briefly explained in Section~\ref{avscore} and in greater detail in~\cite{averitec2024}.
We use the provided \averitec{} scoring script to calculate the values for Table~\ref{tab:pipeline_scores}, using its EvalAI blackbox to obtain the test scores without seeing the gold test data.

The latter experiments shown in Table~\ref{tab:pipeline_scores} suggests the superiority of GPT-4o to predict the results for our pipeline with a margin.
Even if we simplify the evidence \& label generation step by omitting the dynamic few-shot learning (section~\ref{sec:generation}), answer-type tuning and Likert-scale confidence emulation, it still scores above others, also showing that our pipeline can be further simplified when needed.
Regardless of the LLM in use, the results of our pipeline improve upon the \averitec{} baseline dramatically.

Posterior to the original experiments and to the \averitec{} submission deadline, we also compute the pipeline results using an open-source model -- the Llama 3.1 70B\footnote{\url{https://huggingface.co/hugging-quants/Meta-Llama-3.1-70B-Instruct-AWQ-INT4}}~\cite{dubey2024llama3herdmodels} obtaining encouraging scores, signifying our pipeline being adaptable to work well without the need to use a blackboxed proprietary LLM.

\subsection{API costs}
During our experimentation July 2024, we have made around 9000 requests to OpenAI's \texttt{gpt-4o-2024-05-13} batch API, at a total cost of \$363.
This gives a mean cost estimate of \$0.04 per a single fact-check (or \$0.08 using the API without the batch discount) that can be further reduced using cheaper models, such as \texttt{gpt-4o-2024-08-06}.

We argue that such costs make our model suitable for further experiments alongside human fact-checkers whose time spent reading through each source and proposing each evidence by themselves would certainly come at a higher price.

Our successive experiments with Llama 3.1~\cite{dubey2024llama3herdmodels} show promising results as well, nearly achieving parity with GPT.
The use of open-source models such as LLaMa or Mistral allows running our pipeline on premise, without leaking data to a third party and billing anything else than the computational resources.
For further experiments, we are looking to integrate them into the attached Python library using VLLM~\cite{vllm}.

\subsection{Error analysis}
In this section, we provide the results of an explorative analysis of 20 randomly selected samples from the development set. We divide our description of the analysis into the pipeline and dataset errors.


\subsubsection{Pipeline errors}
Our pipeline tends to rely on unofficial (often newspaper) sources rather than official government sources, e.g., with a domain ending or containing \texttt{gov}. On the other hand, it seems that the annotators prefer those sources. This could be remedied by implementing a different source selection strategy, preferring those official sources. For an example, see Listing~\ref{lst:gov_error}.

Another thing that could be recognised as an error is that our pipeline usually generates all ten allowed questions (upper bound given by the task~\cite{averitec2024}). The analysis of the samples shows that the last questions are often unrelated or redundant to the claim and do not contribute directly to better veracity evaluation. However, since the classification step of our pipeline is not dependent on the number of question-answer pairs, this is not a critical error.
Listing~\ref{lst:unrelated_questions} shows an example of a datapoint with some unrelated questions.

When the pipeline generates extractive answers, it sometimes happens that the answer is not precisely extracted from the source text but slightly modified. An example of this error can be seen in Listing~\ref{lst:extractive_error}. This error is not critical, but it could be improved in future works, e.g. using post-processing via string matching.

Individual errors were also caused by the fact that we do not use the claim date in our pipeline and because our pipeline cannot analyse PDFs with tables properly. The last erroneous behaviour we have noticed is that the majority of questions and answers are often generated from a single source. This should not be viewed as an error, but by introducing diversity into the sources, the pipeline would be more reliable when deployed in real-world scenarios.

\subsubsection{Dataset errors}
During the error analysis of our pipeline, we also found some errors in the \averitec{} dataset that we would like to mention. In some cases, there is a leakage of PolitiFact fact-checking articles where the claim is already fact-checked. This leads to a situation where our pipeline gives a correct verdict using the leaked evidence. However, annotators gave a different label (often Not Enough Evidence). 

Another issue we have noticed is the inconsistency in the questions and answers given by annotators. Sometimes, they are long, including non-relevant information, but sometimes, they are at the correct length. The questions are often too general, or the annotators seem to use outside knowledge. This inconsistency in the dataset leads to a decreased performance of any models evaluated on this dataset.

\subsubsection{Summary}
Despite the abovementioned errors, the explorative analysis revealed that our pipeline consistently gives reasonable questions and answers for the claims. Most misclassified samples in those 20 data points were due to dataset errors.


%%!TEX ROOT=../emnlp2023.tex

\section{Python Library}
\label{sec:software}
\todo{write}
To facilitate further experimentation with our pipeline and the reusal of all the tools prompts and software, we have published our code as ready-to-use python library.

The retrieval, evidence generation and veracity inference steps are  \texttt{EvidenceGenerator}
%!TEX ROOT=../emnlp2023.tex

\section{Conclusion}
\label{sec:conclusion}
In this paper, we describe the use and development of a RAG pipeline over real world claims and data scraped from the web for the \averitec{} shared task.
Its main advantage are its simplicity, consisting of just two decoupled modules -- Retriever and an Evidence~\& Label Generator -- and leveraging the trainable parameters of a LLM rather than on complex pipeline engineering.
The LLMs capabilities may further improve in future, making the upgrades of our system trivial.

In section~\ref{sec:system}, we describe the process of adding features to both modules well in an iterative fashion, describing real problems we have encountered and the justifications of their solution, hoping to share our experience on how to make such systems robust and scoring well.
We publish our failed approaches in section~\ref{sec:failed} and the metrics we observed to benchmark our systems in section~\ref{sec:results}. 
We release our Python codebase to facilitate further research and applications of our system, either as a baseline for future research, or for experimenting alongside human fact-checkers.

\subsection{Future works}
\begin{enumerate}
    \item Integrating a search API for use in the wild 
    \item Re-examine the Likert-scale rating (section~\ref{likert}) to establish a more appropriate and fine-grained means of tokenizing the label probabilities
    \item Generating evidence in the form of declarative sentences rather than Question-Answer pairs should be explored to see if it leads for better or worse fact-checking performance
    \item RAG-tuned LLMs such as those introduced in~\cite{menick2022teachinglanguagemodelssupport} could be explored to see if they offer a more reliable source citing
\end{enumerate}
%%%%%%%%%%%%%%%%%%%%%%%%%%%%%%%%%%%%

\section*{Limitations}
The evaluation of our fact-checking pipeline is limited to the English language and the \averitec{} dataset~\cite{averitec2024}. This is a severe limitation as the pipeline when deployed in a real-world application, would encounter other languages and forms of claims not covered by the used dataset.

Another limitation is that we are using a large language model. Because of that, future usage is limited to using an API of a provider of LLMs or having access to a large amount of computational resources, which comes at significant costs. Using APIs also brings the disadvantage of sending data to a third party, which might be a security risk in some critical applications. LLM usage also has an undeniable environmental impact because of the vast amount of electricity and resources used.

The reliability of the generated text is a limitation that is often linked to LLMs. LLMs sometimes hallucinate (in our case, it would mean using sources other than those given in the system prompt), and they can be biased based on their extensive training data. Moreover, because of the dataset size, it is impossible to validate each output of the LLM, and thus, we are not able to 100 \% guarantee the quality of the results.

\section*{Ethics Statement}
It is essential to note that our pipeline is not a real fact-checker that could do a human job but rather a study of future possibilities in automatic fact-checking and a showcase of the current capabilities of state-of-the-art language models. The pipeline in its current state should only be used with human supervision because of the potential biases and errors that could harm the consumers of the output information or persons mentioned in the claims. The pipeline could be misused to spread misinformation by directly using misinformation sources or by intentionally modifying the pipeline in a way that will generate wrong outputs.

Another important statement is that our pipeline was in its current form explicitly built for the \averitec{} shared task, and thus, the evaluation results reflect the bias of the annotators. For more information, see the relevant section of the original paper~\cite{averitec2024}.

The carbon costs of the training and running of our pipeline are considerable and should be taken into account given the urgency of climate change. At the time of deployment, the pipeline should be run on the smallest possible model that can still provide reliable results, and the latest hardware and software optimisations should be used to minimise the carbon footprint.

\section*{Acknowledgements}
We would like to thank Bryce Aaron from UNC for exploring the problems of search query generation and pinpointing claims of underrepresented labels using numerical methods that did not make it into our final pipeline but gave us a frame for comparison. 

This research was co-financed with state support from the Technology Agency of the Czech Republic and the Ministry of Industry and Trade of the Czech Republic under the TREND Programme, project FW10010200.
The access to the computational infrastructure of the OP VVV funded project CZ.02.1.01/0.0/0.0/16\_019/0000765 ``Research Center for Informatics'' is also gratefully acknowledged.
We would like to thank to \mbox{OpenAI} for providing free credit for their paid API via Researcher Access Program\footnote{https://openai.com/form/researcher-access-program/}.



% Entries for the entire Anthology, followed by custom entries
\bibliography{anthology,custom}
\bibliographystyle{acl_natbib}

\appendix

%!TEX ROOT=../emnlp2023.tex


\lstset{
    language={},
    basicstyle=\ttfamily\footnotesize\linespread{0.9}, % Smaller font with less spacing
    keywordstyle=\color{blue}\bfseries,
    commentstyle=\color{green!50!black}\itshape,
    stringstyle=\color{orange},
    numberstyle=\tiny\color{gray},
    numbers=none, % Line numbers on the left
    stepnumber=1, % Line numbers for every line
    numbersep=5pt, % Space between line numbers and code
    tabsize=4, % Size of tabs
    showstringspaces=false, % Don't show spaces in strings
    breaklines=true, % Line wrapping
    breakatwhitespace=true,
    frame=lines, % Add a frame around the code
    captionpos=b, % Caption at the 
    breakindent=1em,
}
\begin{figure*}
    \section{System prompt}
    \label{appendix_sec:system_prompt}
    \begin{lstlisting}[breaklines=true, language={}, frame=single, caption={System prompt for the LLMs, \averitec{} claim is to be entered into the user prompt. Three dots represent omitted repeating parts of the prompt.}, label={lst:llm_system_prompt}]
You are a professional fact checker, formulate up to 10 questions that cover all the facts needed to validate whether the factual statement (in User message) is true, false, uncertain or a matter of opinion. Each question has one of four answer types: Boolean, Extractive, Abstractive and Unanswerable using the provided sources.
After formulating Your questions and their answers using the provided sources, You evaluate the possible veracity verdicts (Supported claim, Refuted claim, Not enough evidence, or Conflicting evidence/Cherrypicking) given your claim and evidence on a Likert scale (1 - Strongly disagree, 2 - Disagree, 3 - Neutral, 4 - Agree, 5 - Strongly agree). Ultimately, you note the single likeliest veracity verdict according to your best knowledge.
The facts must be coming from these sources, please refer them using assigned IDs:
---
## Source ID: 1 [url]
[context before]
[page content]
[context after]
...

---
## Output formatting
Please, you MUST only print the output in the following output format:
```json
{
 "questions":
     [
         {"question": "<Your first question>", "answer": "<The answer to the Your first question>", "source": "<Single numeric source ID backing the answer for Your first question>", "answer_type":"<The type of first answer>"},
         {"question": "<Your second question>", "answer": "<The answer to the Your second question>", "source": "<Single numeric Source ID backing the answer for Your second question>", "answer_type":"<The type of second answer>"}
     ],
 "claim_veracity": {
     "Supported": "<Likert-scale rating of how much You agree with the 'Supported' veracity classification>",
     "Refuted": "<Likert-scale rating of how much You agree with the 'Refuted' veracity classification>",
     "Not Enough Evidence": "<Likert-scale rating of how much You agree with the 'Not Enough Evidence' veracity classification>",
     "Conflicting Evidence/Cherrypicking": "<Likert-scale rating of how much You agree with the 'Conflicting Evidence/Cherrypicking' veracity classification>"
 },
 "veracity_verdict": "<The suggested veracity classification for the claim>"
}
```
---
## Few-shot learning
You have access to the following few-shot learning examples for questions and answers.:

### Question examples for claim "{example["claim"]}" (verdict {example["gold_label"]})
"question": "{question}", "answer": "{answer}", "answer_type": "{answer_type}"
...
    \end{lstlisting}
\end{figure*}
%%!TEX ROOT=../emnlp2023.tex

\section{Open-Source LLMs}
\label{appendix_sec:opensource_llms}
%!TEX ROOT=../emnlp2023.tex

\section{Examples of Errors}
\label{appendix_sec:errors}


\begin{lstlisting}[caption={Example of a claim where our pipeline uses newspaper sources instead of official government sources.}, label={lst:gov_error}, breaklines=true, breakatwhitespace=false, columns=fullflexible]
    Claim 479: Donald Trump said "When the anarchists started ripping down our statues and monuments, right outside, I signed an order immediately, 10 years in prison."

    gold evidence example: 
        question: What was the law signed by Trump regarding damaging federal property?
        answer: Trump signed an executive order that authorizes a penalty of up to 10 years in prison for damaging federal property. It does not say that it will automatically be a 10 year prison sentence.,
        source_url: https://web.archive.org/web/20210224033536/https://www.whitehouse.gov/presidential-actions/executive-order-protecting-american-monuments-memorials-statues-combating-recent-criminal-violence/

    pipeline evidence example: 
        question: Did Trump sign an order related to vandalism of statues and monuments?, 
        answer: Yes, Trump signed an executive order to prosecute those who damage national monuments, making it a punishable offense with up to 10 years in jail.,
        url: https://m.economictimes.com/news/international/world-news/trump-makes-vandalising-national-monuments-punishable-offence-with-up-to-10-yrs-jail/articleshow/76658610.cms
    
\end{lstlisting}

\begin{lstlisting}[caption={Example of a claim and questions showing that the last tends to be unrelated or redundant to fact-checking of the claim.}, label={lst:unrelated_questions}, breaklines=true, breakatwhitespace=false, columns=fullflexible]
    Claim 295: Trump campaign asked Joe Biden to release a list of potential Supreme Court picks only after Ginsburg's passing

    question 1: Did Joe Biden claim that the Trump campaign asked him to release a list of potential Supreme Court picks only after Ginsburg's passing?
    question 2: Did the Trump campaign ask Joe Biden to release a list of potential Supreme Court picks before Ginsburg's passing?
    question 3: When did Trump release his latest list of potential Supreme Court nominees?
    question 4: Did Trump personally demand that Biden release a list of potential Supreme Court nominees before Ginsburg's death?
    question 5: What did Trump say about Biden releasing a list of potential Supreme Court nominees during the Republican National Convention?
    question 6: Did the Trump campaign issue a statement on September 17, 2020, regarding Biden releasing a list of potential Supreme Court nominees?
    question 7: What did the Trump campaign's statement on September 9, 2020, say about Biden releasing a list of potential Supreme Court nominees?
    question 8: Did Biden indicate in June 2020 that he might release a list of potential Supreme Court picks?
    quetion 9: What reason did Biden give for not releasing a list of potential Supreme Court nominees?,
    question 10: Did Biden pledge to nominate a Black woman to the Supreme Court?
    
\end{lstlisting}

\begin{lstlisting}[caption={Example of a claim where our pipeline did not exactly extract the answer.}, label={lst:extractive_error}, breaklines=true, breakatwhitespace=false, columns=fullflexible]
    Claim #155 - Trump said 'there were fine people on both side' in far-right protests.

    answer: "You had some very bad people in that group, but you also had people that were very fine people, on both sides.", 
    answer_type: Extractive
    url: https://www.theatlantic.com/politics/archive/2017/08/trump-defends-white-nationalist-protesters-some-very-fine-people-on-both-sides/537012/

    scraped text: ... "You also had some very fine people on both sides," he said. The Unite the Right rally that sparked the violence in Charlottesville featured several leading names in the white-nationalist alt-right movement, and also attracted people displaying Nazi symbols. ...
    
\end{lstlisting}

\end{document}
