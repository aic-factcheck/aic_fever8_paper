%!TEX ROOT=../emnlp2023.tex

\section{Results and analysis}
\label{sec:results}
\todo{write}

\begin{table*}[h]
    \centering
    \begin{tabular}{l | c c c | c c c}
    \hline
    &\multicolumn{3}{c|}{\textbf{Dev Set Scores}} & \multicolumn{3}{c}{\textbf{Test Set Scores}}  \\
    \textbf{Pipeline Name} & \textbf{Q only} & \textbf{Q+A} & \textbf{\averitec{}} & \textbf{Q only} & \textbf{Q+A} & \textbf{AVeriTeC} \\ \hline

    
    \textbf{GPT-4o (full-featured pipeline)}      & \textbf{0.46} & \textbf{0.29} & \textbf{0.42} & \textbf{0.46} & \textbf{0.32} & \textbf{0.50}\\
    GPT-4o (simplified pipeline)         & 0.45 & 0.28 & 0.38 & 0.45 & 0.30 & 0.47 \\
    Claude-3.5-Sonnet (full-featured)             & 0.43 & 0.28 & 0.35 & 0.42 & 0.30 & 0.46 \\
    GPT-4o (with DeBERTa classification)              & 0.45 & 0.28 & 0.36 & -- & -- & --\\
    \averitec{} baseline            & 0.24 & 0.19 & 0.09 & 0.24 & 0.20 & 0.11\\
    \hline
    Llama 3.1 70B (full-featured) & 0.46 & 0.27 & 0.36 & 0.47 & 0.29 & 0.42\\
    \bottomrule
    \end{tabular}
    \caption{Comparison of Pipeline Scores on Dev and Test Sets, AVeriTeC scores are @0.25}
    \label{tab:pipeline_scores}
\end{table*}
    

\subsection{API Costs}
During our experimentation July 2024, we have made around 9000 requests to OpenAI's \texttt{gpt-4o-2024-05-13} batch API, at a total cost of \$363.
This gives a mean cost estimate of \$0.04 per a single fact-check (or \$0.08 using the API without the batch discount) that can be further reduced using cheaper models, such as \texttt{gpt-4o-2024-08-06}.

We argue that such costs make our model suitable for further experiments alongside human fact-checkers whose time spent reading through each source and proposing each evidence by themselves would certainly come at a higher price.

Our successive experiments with LLaMa 3.1~\cite{meta2024llama31} show promising results as well, nearly achieving parity with GPT.
The use of open-source models such as LLaMa or Mistral allows running our pipeline on premise, without leaking data to a third party and billing anything else than the computational resources.
For further experiments, we are looking to integrate them into the attached Python library using VLLM~\cite{vllm}.

\subsection{Strenghts \& Error analysis}
In this section we provide results of an explorative analysis of 20 randomly selected samples from the development set. We divide our analysis into three parts: pipeline strenghts and errors, and dataset errors which we came across during our reviewing of the samples.

\subsubsection*{Strenghts}
\todo{}
\subsubsection*{Pipeline Errors}
\todo{používání neoficiálních zdrojů ("gov" chyba), otázky navíc (udělali jsme 10) }
\subsubsection*{Dataset Errors}

\todo{what with this?}
All of the classifiers and pipelines we have set up for our task have exposed a significant dip in $F_1$ label score for the \texttt{Not Enough Evidence} and \texttt{Conflicting Evidence/Cherrypicking} labels.


